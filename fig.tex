\documentclass[cropped]{standalone}

\usepackage{tikz}

% helvetica font
\usepackage[scaled=1.1]{helvet}
\usepackage{sfmath}
\renewcommand{\familydefault}{\sfdefault}

% or comment out previous 3 lines, and uncomment the one below, for clean latex font
%\usepackage{cmbright}

% you might want to define some colors:
\definecolor{exc}{HTML}{E51E10}
\definecolor{inh}{HTML}{56B4E9}
\definecolor{control}{HTML}{41AB5D}
\definecolor{m1}{HTML}{D73027}
\definecolor{m2}{HTML}{F46D43}
\definecolor{m3}{HTML}{FDAE61}
\definecolor{m4}{HTML}{FEE090}
\definecolor{m5}{HTML}{ABD9E9}
\definecolor{m6}{HTML}{74ADD1}
\definecolor{m7}{HTML}{4575B4}
\definecolor{m8}{HTML}{313695}
 
\usetikzlibrary[positioning,arrows,arrows.meta,shapes,calc,spy,intersections,shadows,bending]
\usetikzlibrary[decorations.text,decorations.pathmorphing,decorations.markings]

\pagestyle{empty}
\thispagestyle{empty}

\begin{document}

\begin{tikzpicture}

% ----------------------------------------------------

% helper grid (comment out once you're done!)
%\draw[help lines,step=1,thin,gray!20] (0,0) grid (100, 100);
%\draw[help lines,step=5,gray!20,very thick] (0,0) grid (100, 100);
%\draw[help lines,step=10,gray!40,ultra thick] (0,0) grid (100, 100);
%
%\foreach \j in {10,15,..., 50}
%  \node at (\j,85) [fill=white,text=red!80!black] {\j};
%\foreach \j in {10,15,..., 95}
%  \node at (23,\j) [fill=white,text=green!80!black] {\j};

% ----------------------------------------------------

% plot 8 straight movements 
\node (c) at (37.9,82.7) [circle,inner sep=0em,minimum width=4.5em] {};
\draw [m1,very thick] (c.center) -- (c.-36);
\draw [m2,very thick] (c.center) -- (c.0);
\draw [m3,very thick] (c.center) -- (c.36);
\draw [m4,very thick] (c.center) -- (c.72);
\draw [m5,very thick] (c.center) -- (c.108);
\draw [m6,very thick] (c.center) -- (c.144);
\draw [m7,very thick] (c.center) -- (c.180);
\draw [m8,very thick] (c.center) -- (c.216);
 

% ----------------------------------------------------

\node at (0,0) [above right,inner sep=0em] {\includegraphics{_tmp.pdf}};

% HERE YOU CAN USE TIKZ CODE TO ADD DIAGRAMS ETC

% ------------------------------------------------------------------------
% ---      SOC schematics                                              ---
% ------------------------------------------------------------------------
 
\node (soc0) at (35,82) [circle,inner sep=0em,minimum width=8em] {};

\begin{scope}[control,thick]
\node (c1) at (soc0.west) [left=3em,align=right] {state\\initialization};
\foreach \j in {160,170,...,200} 
  \draw [->,>=latex] (c1.east) .. controls ++(0.4,0) .. (soc0.\j);
\end{scope}

\node (arm) at (soc0.east) [above right=-2.5em and 0.5em,circle,inner sep=0em,minimum width=8em,rotate=-20] 
  {\includegraphics[width=4em]{truncated_body.png}};

\coordinate (tout) at ($(soc0.east)+(1.2,0)$);
\begin{scope}[thick]
\foreach \j in {-20,-10}
  \draw (tout) .. controls ++(-0.4,0) .. (soc0.\j);
\foreach \j in {10,20}
  \draw (tout) .. controls ++(-0.4,0) .. (soc0.\j);
\draw [<-,>=latex] ($(tout)+(0.1,0)$) .. controls ++(-0.5,0) .. (soc0.0);
\end{scope}

\begin{scope}[shift={+(11,-13)}]
\coordinate (a0) at (26.5,94);
\coordinate (a1) at (27.1,94);
\coordinate (a2) at (27.7,94.5);
\coordinate (a3) at (27,95.6);
\draw [white,line width=2.3pt,line cap=round] (a0) to (a1) to (a2) to (a3);
\draw [black,dotted,thick] (a0) to (a1);
\draw [black,thick] (a1) to (a2);
\draw [black,thick] (a2) to (a3);
\fill [black] (a1) circle (0.25em);
\fill [white] (a1) circle (0.17em);
\fill [black] (a2) circle (0.25em);
\fill [white] (a2) circle (0.17em);
\end{scope}

% soc with E and I neurons
\node (soc) at (soc0) [circle,thick,draw=black,fill=white,inner sep=0em,minimum width=8em] {};
\node at (soc.north) [below=0.9em,font=\bfseries] {M1};
\begin{scope}[shift={(soc0)},x={(4em,0em)},y={(0em,4em)}]
\node (e1) [inner sep=0em,minimum width=0.05em] at (0.6,-0.2) {};
\node (e2) [inner sep=0em,minimum width=0.05em] at (0.1,-0.6) {};
\node (e3) [inner sep=0em,minimum width=0.05em] at (-0.5,0.2) {};
\node (i1) [inner sep=0em,minimum width=0.05em] at (0.1,0.15) {};
\node (i2) [inner sep=0em,minimum width=0.05em] at (-0.4,-0.45) {};
\draw [-*,thick,exc,bend left] (e1) to (e2);
\draw [-*,thick,exc,bend left] (e2) to (e1);
\draw [-*,thick,exc,bend left] (e2) to (i2);
\draw [-*,thick,exc,bend right] (e1) to (i1);
\draw [-*,thick,exc,bend left] (e3) to (i1);
\draw [-*,thick,exc,bend right] (e3) to (i2);
\draw [-*,thick,exc,bend right] (e2) to (i1);
\draw [-*,thick,exc,bend left] (e1) to (e3);
\draw [-*,thick,inh,bend left] (i1) to (e3);
\draw [-*,thick,inh,bend right] (i1) to (i2);
\foreach \j in {1,2,3} \node [fill=exc!40,thick,draw=exc,circle,inner sep=0em,minimum width=0.7em] at (e\j) {};
\foreach \j in {1,2} \node [fill=inh!40,thick,draw=inh,circle,inner sep=0em,minimum width=0.7em] at (i\j) {};
\end{scope}


% add label for E/I inputs

\node at (41.5, 79) [rotate=90,anchor=base] {\textcolor{exc}{exc}/\textcolor{inh}{inh} inputs (z-score)};

% don't forget all your panel labels etc
% e.g.:
\node at (29.1,83) [anchor=base,font=\bfseries,scale=1.5] {A};
\node at (29.1,79.6) [anchor=base,font=\bfseries,scale=1.5] {B};
\node at (34.7,79.6) [anchor=base,font=\bfseries,scale=1.5] {C};
\node at (29.1,76) [anchor=base,font=\bfseries,scale=1.5] {D};
\node at (41.3,83) [anchor=base,font=\bfseries,scale=1.5] {E};
\node at (29.1,72.5) [anchor=base,font=\bfseries,scale=1.5] {F};


% figure caption!

\node at (34, 73.3) [below right,scale=1.3,text width=39em,align=justify] {
  \baselineskip=14pt
  {\bfseries FIGURE 1} ---
  {\bfseries (A)}~An inhibition-stabilized network (ISN) model of M1
    is initialized in movement-dependent states, from which dynamics
    evolve and result in the production of two torques actuating
    a two-link arm. The model is calibrated so as to produce 8 different
    straight reaches.
  {\bfseries (B)}~Connectivity matrix of the ISN, showing sparse random
  excitatory connections and denser inhibitory ones optimized for robust stability.
  {\bfseries (C)}~Eigenvalue spectrum of the connectivity matrix shown in (B). Note
  the presence of many eigenvalue pairs with poor damping, indicative of strong
  oscillations in the corresponding eigenplanes.
  {\bfseries (D)}~Example single-neuron responses to each of the 8 straight reaches
  (cf.\ color code in panel A). 
  {\bfseries (E)}~Normalized total recurrent E (red) and I (blue) inputs, 
  for three example neurons in the model (horizontally), at three different times
  during movement (vertically; early, mid-movement, and late).
  For each neuron, the Pearson correlation between E and I inputs across the 8 movements
  is calculated, and population histograms are shown on the right.
  {\bfseries (F)}~Top: power-spectra of three individual sample neurons in the model,
  during spontaneous (noise-driven) activity. Note the absence of well-defined oscillations.
  Middle: power-spectra of the contributions of each eigenplane to the momentary activity
  (colors only aid visualization; not all eigenplanes are shown for clarity).
  Bottom: power-spectra of the activity projections recovered by our new analysis technique
  (cf.\ text).
  \par
};




\end{tikzpicture}
\end{document}

